\documentclass{article}
\usepackage[utf8]{inputenc}
\usepackage[swedish]{babel}
\usepackage[margin=3cm, a4paper]{geometry}
\usepackage{newcent}
\usepackage{graphicx}
\graphicspath{ {./graphics/} }
\usepackage{wrapfig}
\usepackage{color}
\usepackage{hyperref}
\usepackage{parskip}
\usepackage{subfiles}
\usepackage{tabularx}
\usepackage{makecell}
\renewcommand{\thefootnote}{\alph{footnote}}
\renewcommand{\thesection}{\S\arabic{section}}

\newcommand{\documentdate}{2019-XX-XX}

\usepackage{fancyhdr}
\pagestyle{fancy}
\fancyhf{}
\rhead{Linköping\\\documentdate}
\lhead{Stadgar\\Förenade Individer för Kulinariska Arrangemang}
\rfoot{\thepage}

\title{Förenade Individer för Kulinariska Arrangemang \\
                F.I.K.A \\
                Stadgar}
\author{}
\date{\documentdate}

\begin{document}
\pagenumbering{roman}
\maketitle
\newpage

\subsection*{Versionshistorik}

\begin{tabularx}{\textwidth}{|l|l|X|l|l|l|}
\hline
Version & Datum & Anmärkning & \makecell{Första\\läsning} &
\makecell{Andra\\läsning} & Ansvarig \\
\hline
\end{tabularx}
\newpage

\tableofcontents
\newpage
\pagenumbering{arabic}

\section{Ändamål}
\subsection{Syfte}
Föreningens syfte är att samla personer som uppskattar den svenska
fikatraditionen och som vill fika med likasinnade människor.
\subsection{Namn}
Föreningens namn är Förenade Individer för Kulinariska Arrangemang.
\subsubsection{Förkortning}
Föreningens namn förkortas F.I.K.A. eller FIKA.
\subsection{Säte}
Förenade Individer för Kulinariska Arrangemang har sitt säte i Linköping.


\section{Medlemskap}
Medlemskap erhålles genom inskrivande i föreningens register.
\subsection{Ansökan}
För att bli medlem i föreningen ska personen i fråga ansöka till
medlemsfunktionären. Medlemsfunktionären skriver in personen i ett tillfälligt
register och skriver in denne i det officiella registret först när en
inkilningsfika har bjudits på av personen sökande medlemskap.
\subsection{Skyldighet}
Som medlem i föreningen har medlemmen skyldighet att bjuda en del av föreningens
medlemmar på fika i enlighet med \ref{fika} och dess underparagrafer. Detta görs
med medlemmens egna kapital och förmåga.

\subsection{Inkilningsfika}
För att en person ska bli officiellt inskriven i registret över medlemmar krävs
det att personen bjuder på fika i enlighet med \ref{fikafika},
\ref{fikastandard}, och \ref{fredagsfika}. Ett inkilningsfika ska bjudas utanför
schemalagda tillfällen, där ett undantag finns för fredagsfika, se
\ref{fredagsfika}.

\subsection{Utträde}
En föreningsmedlem som ej längre önskar vara medlem i föreningen kan begära
utträde. Detta sker skriftligen till medlemsfunktionären som ska behandla
frågan skyndsamt.

\section{Verksamhet}
\subsection{Organisation}
\begin{itemize}
  \item Föreningens föreningsmöte
  \item Föreningens styrelse
\end{itemize}
\subsection{Verksamhetsår}
Föreningens verksamhetsår sträcker sig från den 1 augusti till den 30 juli.

\section{Föreningsmöte}
\subsection{Syfte}
Föreningsmötet är föreningens högst beslutande organ.
\subsection{Verksamhet}
Föreningsmötet har att verkställa erforderliga val, granska styrelsens och
övriga organs verksamhet, dra upp riktlinjer för verksamheten i stort samt
fastställa den ekonomiska ramen för verksamheten.
\subsection{Sammanträde}
Föreningsmötet sammanträder i början av varje kalenderår.
\subsection{Kallelse}
Kallelse till föreningsmötet ska göras tillgänglig till föreningens medlemmar
senast 14 dagar innan föreningsmötet.
\subsection{Propositioner och motioner}
Propositioner och motioner måste skriftligen inkomma till föreningsstyrelsen
senast tio dagar före föreningsmötet.
\subsection{Extra föreningsmöte}
Extra föreningsmöte kan kallas av föreningsstyrelsen eller av en grupp bestående
av minst 15 föreningsmedlemmar. Ett extra föreningsmöte kan ej inträffa inom
30 dagar från senaste föreningsmöte.

\section{Föreningsstyrelsen}
\subsection{Organisation}
Styrelsen består av följande ordinarie medlemmar:
\begin{itemize}
  \item Ordförande
  \item Kassör
  \item Medlemsfunktionär
\end{itemize}
Styrelsen kan bestå utav flera medlemmar än de specificerade ifall
föreningsmötet har beslutat det.

\subsection{Sammanträde}

\subsection{Beslutsmässighet}

\subsection{Beslutsmöte}

\subsection{Ordförande}

\subsection{Kassör}

\subsection{Medlemsfunktionär}

\subsection{Firmateckning}

\section{Fika} \label{fika}
\subsection{Fika} \label{fikafika} %TODO rewrite
Fika definieras i detta dokument som ett sött tilltugg till kaffe eller liknande
dryck där tilltugget tillhandahålls av den fikaansvarige. Dryck är inte
obligatoriskt för den fikaansvarige att tillhandahålla, detta kan dock väga upp
för sämre tilltugg sådan att den fikaansvariges fika blir godkänd av den fikande
gruppen. En fika är inte en måltid och en fika får inte bara bestå av dryck.
Fikat ska räcka till varje medlem i det fikande sällskapet. Undantaget till alla
dessa regler är brist på kapital.

\subsection{Standardfika} \label{fikastandard} %TODO rewrite
En standardfika definieras som ett tilltugg som får plats och fyller en öppnad
hand till stor grad. Torra kex och liknande tilltugg räknas inte som fika, om
inte något annat också erbjuds, såsom kaffe, saft, ost, eller liknande. Undantag
för regler kan göras vid brist på kapital.

\subsection{Hembakat} %TODO rewrite
Hembakat är alltid bättre än köpt. Detta gör att det går att undkomma regeln för
storlek av fika, om det är så att det erbjudna tilltugget är hembakat.

\subsection{Fredagsfika} \label{fredagsfika} %TODO rewrite
Varje fredag på schemalagda veckor på Tekniska högskolan vid Linköpings
universitet bör ett fredagsfika anordnas av en föreningsmedlem. Fredagsfikats
anordnande sker på ett rullande schema i inskriven ordning. Närvaro är inte
obligatorisk för dem som inte bjuder på fika, obligatorisk närvaro gäller dock
för medlemmen som bjuder på fika. Om medlemmen som ska bjuda på fika inte kan
närvara ska den medlemmen ta kontakt med nästa medlem på listan så att fika
alltid erbjuds på fredagar, samt att sällskapet är informerat. Fredagsfikat sker
med fördel på lunchen, men detta kan ändras om sällskapet informeras samt
godkänns av gruppen. Under tenta- och omtenta-p kan Fredagsfika erbjudas även
fast det inte är fredag, då fika behövs under dessa perioder.

En fredagsfika kan tillgodoräknas som en inkilningsfika om detta görs utöver
schemabunden fredagsfika eller under tenta-p.

\subsection{Eftertentafika} %TODO rewrite
Efter Tenta-P skall ett stort, gemensamt fika anordnas, ett s.k.
Efter-tenta-fika. Under ett sådant fika skall varje medverkande individ i
sällskapet ta med sig något för att bidra till efter-tenta-fikat.

\subsubsection{Omfika}
På kvällen efter ett tentafika anordnas ett omfika för de som missade
eftertentafikat , eller de medlemmar som vill fika mer. Varje omfika skall vara
temabunden enligt \ref{omfikateman} i bästa möjliga mån.

\subsubsection{Omfikateman} \label{omfikateman}
Siffrorna i listan representerar tenta-/omtenta-p i läsårsordning.
\begin{enumerate}
  \item Förfest till tentakravaller
  \item TV-spelskväll
  \item Smörgåsbordsfest
  \item Stadsvandring
\end{enumerate}

\section{Insignia}

\section{Ekonomi}

\section{Stadgar}
\subsection{Stadgeändring}
Förslag till ändringar av dessa stadgar skall skriftligen inlämnas till
sektionens styrelse minst tio läsdagar före det föreningsmöte vid vilket
förslaget önskas behandlat. Ändringsförslaget skall omedelbart anslås av
styrelsen.
\subsection{Beslut}
För beslut fordras att  minst 2/3 av antalet närvarande mötesdeltagare är ense
om beslutet.

Beslutet träder i kraft vid mötets avslut.
\subsection{Tolkning}
Vid frågor om tolkning av dessa stadgar gäller styrelsens mening, tills frågan
avgjorts av föreningsmötet.

\section{Upplösning}
\subsection{Kvarvarande medel}
\subsection{Likvidering}

\end{document}
