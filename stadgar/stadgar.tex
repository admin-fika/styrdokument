\documentclass{article}
\usepackage[utf8]{inputenc}
\usepackage[swedish]{babel}
\usepackage[margin=3cm, a4paper]{geometry}
\usepackage{newcent}
\usepackage{graphicx}
\graphicspath{ {./graphics/} }
\usepackage{wrapfig}
\usepackage{color}
\usepackage{hyperref}
\usepackage{parskip}
\usepackage{subfiles}
\usepackage{tabularx}
\usepackage{makecell}
\renewcommand{\thefootnote}{\alph{footnote}}
\renewcommand{\thesection}{\S\arabic{section}}

\title{Förenade Individer för Kulinariska Arrangemang \\
                F.I.K.A \\
                Stadgar}
\author{Föreningen F.I.K.A}
\date{2019-11-XX}

\begin{document}
\pagenumbering{roman}
\maketitle
\newpage

\subsection*{Versionshistorik}

\begin{tabularx}{\textwidth}{|l|l|X|l|l|l|}
\hline
Version & Datum & Anmärkning & \makecell{Första\\läsning} & \makecell{Andra\\läsning} & Ansvarig \\
\hline
\end{tabularx}
\newpage

\tableofcontents
\newpage
\pagenumbering{arabic}

\section{Ändamål}
\subsection{Syfte}

\section{Medlemskap}
\subsection{Ansökan}
För att bli medlem i föreningen ska personen i fråga ansöka till
medlemsfunktionären. Medlemsfunktionären skriver in personen i ett tillfälligt
register, och skriver in denne i det officiella registret först när en
inkilningsfika har bjudits på av personen.

\subsection{Inkilningsfika}
För att en person ska bli officiellt inskriven i registret över medlemmar krävs
det att personen bjuder på fika i enlighet med \ref{fikafika},
\ref{fikastandard}, och \ref{fredagsfika}.

\subsection{Utträdande}

\section{Verksamhet}
\subsection{Verksamhetsår}

\section{Föreningsstyrelsen}
\subsection{Organisation}
Styrelsen består av följande ordinarie medlemmar:
\begin{itemize}
  \item Ordförande
  \item Medlemsfunktionär
\end{itemize}
Styrelsen kan bestå utav flera medlemmar än de specificerade ifall årsmötet har
beslutat det.

\subsection{Sammanträde}

\subsection{Beslutsmässighet}

\subsection{Beslutsmöte}

\subsection{Ordförande}

\subsection{Medlemsfunktionär}


\subsection{Firmateckning}

\section{Fika}
\subsection{Fika} \label{fikafika}
Fika definieras i detta dokument som ett sött tilltugg till kaffe eller liknande
dryck där tilltugget tillhandahålls av den fikaansvarige. Dryck är inte
obligatoriskt för den fikaansvarige att tillhandahålla, detta kan dock väga upp
för sämre tilltugg sådan att den fikaansvariges fika blir godkänd av den fikande
gruppen. En fika är inte en måltid och en fika får inte bara bestå av dryck.
Fikat ska räcka till varje medlem i det fikande sällskapet. Undantaget till alla
dessa regler är brist på kapital.

\subsection{Standardfika} \label{fikastandard}
En standardfika definieras som ett tilltugg som får plats och fyller en öppnad
hand till stor grad. Torra kex och liknande tilltugg (Mariekex och Digestive
m.m.) räknas inte som fika, om inte något annat också erbjuds, såsom kaffe,
saft, ost, eller liknande. Undantag för regler kan göras vid brist på kapital.

\subsection{Hembakat}
Hembakat är alltid bättre än köpt. Detta gör att det går att undkomma regeln för
storlek av fika, om det är så att det erbjudna tilltugget är hembakat.

\subsection{Fredagsfika} \label{fredagsfika}
Fredagsfika sker på ett rullande schema i inskriven ordning. Närvaro är inte
obligatorisk för de som inte bjuder på fika, detta gäller dock för personen som
bjuder på fika. Om personen som ska bjuda på fika inte kan närvara ska den
personen ta kontakt med nästa person på listan så att fika alltid erbjuds på
fredagar, samt att sällskapet är informerat. Fredagsfikat sker med fördel på
lunchen, men detta kan ändras om sällskapet informeras samt godkänns av gruppen.
Under Tenta- och Omtenta-P kan Fredagsfika erbjudas även fast det inte är
fredag, då fika behövs under dessa perioder.

En fredagsfika kan tillgodoräknas som en inkilningsfika om detta görs utöver
schemabunden fredagsfika eller under tenta-p.

\subsection{Eftertentafika}
Efter Tenta-P skall ett stort, gemensamt fika anordnas, ett s.k.
Efter-tenta-fika. Under ett sådant fika skall varje medverkande individ i
sällskapet ta med sig något för att bidra till efter-tenta-fika:t, där
sällskapet gemensamt bestämmer vad som är tillräckligt.

På kvällen efter ett tentafika anordnas ett omfika för de som missade
eftertentafikat (eller vill fika mer). Varje omfika skall vara temabunden enligt
nedanstående lista, enligt bästa möjliga mån (där siffra representerar
tenta-/omtenta-p).

Omfikateman:
\begin{enumerate}
  \item Förfest till tentakravaller
  \item TV-spelskväll
  \item Smörgåsbordsfest
  \item Stadsvandring
\end{enumerate}

\section{Insignia}

\section{Ekonomi}

\section{Stadgar}
\subsection{Stadgeändring}
\subsection{Beslut}
\subsection{Tolkning}

\section{Upplösning}
\subsection{Kvarvarande medel}
\subsection{Likvidering}

\end{document}
